\documentclass{article}
\usepackage{graphicx} % Required for inserting images
\usepackage{amsmath}

\title{}
\author{Vasilis Georgiou}

\date{November $2024$}
\begin{document}
\maketitle

\section{Datasets}
Datasets are split in REAL or SYNTHETIC and are based on the 
the source of the routing table:
\begin{itemize}
    \item REAL-Tier-1-A: real core backbone router in a global tier-1 ISP
    \item REAL-Tier-1-B: national backbone router in research and educational network of WIDE Project.
    \item SYN1: each procedure that is no longer than /24 and /16 is split
    into two and four prefixes
    \item SYN2: Each prefix that is no longer than /24, /20 and /16 is split into two, four and eight prefixes
\end{itemize}

\noindent
\textbf{Notation:}
\begin{itemize}
    \item binary radix depth: longest prefix matching
\end{itemize}

\section{Traffic Patterns}
Take the following traffic patterns into consideration:
\begin{itemize}
    \item random: $2^{32}$ random traffic patterns (generated by xorshift).
    \begin{itemize}
        \item overhead for generating the data is included in the measurements, 
        but it is small.
    \end{itemize}
    \item sequential: 
    \begin{itemize}
        \item Generates in the range 0.0.0.0 to 255.255.255.255
    \end{itemize}
    \item repeated
        \begin{itemize}
            \item Similar to random but each random lookup is repeated $16$ times
        \end{itemize}
    \item real-trace
    \begin{itemize}
        \item real traffic trace
    \end{itemize}
\end{itemize}

\section{Using the different libraries}
\begin{itemize}
    \item modified\_poptrie
    \begin{itemize}
        \item need to copy the test data files to modified\_poptrie/build/tests
        \item last test doesn't pass
    \end{itemize}
    \item modified\_radix\_tree
    \item modified\_tree\_bitmap
    \begin{itemize}
        \item rm\_test\_v6 \underline{need to pass the input file} here to measure runtimes.
        \item use the runtime information and memory usage to compute the lookup rate
    \end{itemize}
    \item modified\_sail
    \begin{itemize}
        \item uncomment runtime measurement commands in function sailPerformanceTest()
        \item QueryPerformanceFrequency() function doesn't exist
        \item QueryPerformanceCounter() doesn't exist
    \end{itemize}
\end{itemize}

\subsection{Datasets included in libraries}
\begin{itemize}
    \item modified\_poptrie/tests
    \begin{itemize}
        \item linx-rib.20141217.0000-p46.txt
        \item linx-rib.20141217.0000-p52.txt
        \item linx-rib-ipv6.20141225.0000.p69.txt
        \item linx-update.20141217.0000-p52.txt
    \end{itemize}
\end{itemize}

\subsection{Comments on preallocation of memory for data-structures}
\begin{itemize}
    \item It makes sense to preallocate, because the data
    structures will be initialized once and then remain as is.
    \item lookup rate = number of lookups / total runtime
\end{itemize}

\newpage
\subsection{Tables}

\begin{table}
    \begin{center}
        \begin{tabular}{|l|c|c|c|c|c|c|c|}
            \hline
            Configuration   & $s$ & \# inodes & \# leaves & Mem (MiB) & Init (s) & Rate (Mlps) & CPU cycles \\ 
            \hline
            Radix           & -   & &  & & & & \\ 
            % \hline                           
            % Poptrie-basic   & $0$  & & & & & & \\
                            % & $16$ & & & & & & \\
                            % & $18$ & & & & & & \\
            % \hline                           
            % Poptrie-leafvec & $0$ & & & & & & \\
                            % & $16$ & & & & & & \\
                            % & $18$ & & & & & & \\
                            % 124575
            \hline
            Poptrie         & $2$ & $36,412$ & $63,527$ & $7.575$ & $2.16$ & $71.91$ & $45$ \\  
                            & $16$ & $14,664$ & $56,367$ &  & $1.79$ & $285,7$ & \\  
                            & $18$ & $14,664$ & $56,367$ & $6.12$ & $1.80$ & $316.8$ & \\  
            \hline
        \end{tabular}
    \end{center}
    \caption{The compilation time, the number of nodes, the memory footprint, and the lookup rate for
random with direct pointing (s = 0, 16, 18)}
\end{table}
\noindent
\begin{table}
    \begin{center}
        \begin{tabular}{|l|c|c|c|c|c|c|}
            \hline
            Configuration  & $s$ & total memory & routes & trie memory & loads & stores \\
            \hline
            Poptrie  & $2$ & $94.6$ & $55.5$ & $39.1$ & $7.450$ & $0.124$ \\
              & $16$ &  & routes & trie memory & loads & stores \\
              & $18$ &  & routes & trie memory & loads & stores \\
            \hline
        \end{tabular}
    \end{center}
    \caption{The total allocated memory, memory used for the routing table, memory footprint, load accesses, store accesses for poptrie with
 leaf compression, direct pointing and s = 0, 16, 18 (MiB)}
\end{table}
\begin{figure}
\begin{center}
    \includegraphics[scale=0.26]{figures/massif_poptrie_s2.png}
\end{center}
\end{figure}
% \begin{table}
%     \begin{center}
%         \begin{tabular}{|l|c|c|}
%             \hline
%             Configuration  & direct-ptr & alt-ptr \\
%             \hline
%         \end{tabular}
%     \end{center}
% \end{table}
\textbf{characteristics:}
\begin{itemize}
    \item poptrie-basic-s
    \item poptrie-leafvec-s
    \item poptrie-s (direct pointing)
    \item \#inodes
    \item \#leaves
    \item memory footprint
    \item compilation time (we measure this because
    we reconstruct the tree. Does this include rebuilding the tree?)
\end{itemize}

\subsection{Poptrie specification}
\begin{itemize}
    \item FIB: forwarding information base
    \item RIB: routing information base
\end{itemize}

\end{document}